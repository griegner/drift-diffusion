\documentclass[9pt]{article}

\usepackage{subfiles} % include subfiles
\usepackage{fullpage} % 1-in margins
\usepackage{amsmath, amsfonts, amssymb} % equations
\usepackage{graphicx} % figures
\usepackage{subcaption} % subcaptions
\usepackage{float} % figures
\usepackage{xcolor} % links
\usepackage[colorlinks=true, linkcolor=gray, citecolor=gray, urlcolor=gray]{hyperref} % links
\usepackage{natbib} % citations + references
\usepackage[T1]{fontenc} % matplotlib font
\usepackage{dejavu}
\renewcommand*\familydefault{\sfdefault}
\usepackage{titlesec} % sections/subsections/subsubsections
\titleformat{\section}{\large}{\thesection}{1em}{}
\titleformat{\subsection}{\itshape\large}{\thesubsection}{1em}{}
\titleformat{\subsubsection}{\itshape\normalsize}{\thesubsubsection}{1em}{}


\title{\Large Quantifying uncertainty in drift diffusion models of decision making}
\author{Gabriel Riegner $^{1}$, Pamela Reinagel $^{2}$, Armin Schwartzman $^{1,3}$\\
{\small $^{1}$Halicio\u{g}lu Data Science Institute, University of California San Diego}\\
{\small $^{2}$Department of Neurobiology, University of California San Diego}\\
{\small $^{3}$Division of Biostatistics, University of California San Diego}\\
}
\date{}

\begin{document}
\maketitle

\section*{Abstract}
Drift diffusion models (DDMs) are central to modeling decision-making from reaction time and accuracy data. However, standard parameter estimation tools such as HDDM (Bayesian) and DMAT (frequentist MLE) assume trials are independent and identically distributed (i.i.d.), leading to biased uncertainty estimates when real behavioral data exhibit autocorrelation or heteroscedasticity. Furthermore, these methods do not provide principled standard errors for decision making parameters or reaction time distributions.\\

\noindent Here, we present robust asymptotic covariance estimators for MLE-fitted DDMs that remain consistent and asymptotically normal under model misspecification and autocorrelated trials. We further apply the delta method to propagate parameter uncertainties to the DDM density function, enabling confidence intervals for predicted reaction time distributions.\\

\noindent We validate our approach using simulated decision data under i.i.d., misspecified i.i.d., and non-i.i.d. (autocorrelated) conditions, demonstrating unbiased covariance recovery. Application to decision-making data from freely behaving rats reveals ..., uncovering parameter non-stationarity missed by standard approaches.\\

\noindent Our framework enables uncertainty quantification for DDM parameters in realistic, temporally structured datasets, supporting more reliable inference in neuroscience decision-making research.


\section*{Disclosure Statement}
The authors declare no conflicts of interest.

\section*{Funding Sources}
...

\vfill
\noindent\textit{Keywords:} ...

\end{document}